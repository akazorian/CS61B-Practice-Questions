% --------------------------------------------------------------
% This is all preamble stuff that you don't have to worry about.
% Head down to where it says "Start here"
% --------------------------------------------------------------

\documentclass{article}

\usepackage{amsmath,amsthm,amssymb,listings}
\usepackage[utf8]{inputenc}
\usepackage{geometry}
\geometry{margin=2cm} % Use the Adobe Utopia font for the document - comment this line to return to the LaTeX default
\usepackage[english]{babel} % English language/hyphenation

\usepackage{sectsty} % Allows customizing section commands
\allsectionsfont{\centering \normalfont\scshape} % Make all sections centered, the default font and small caps

\usepackage{fancyhdr} % Custom headers and footers
\pagestyle{fancyplain} % Makes all pages in the document conform to the custom headers and footers
\fancyhead{} % No page header - if you want one, create it in the same way as the footers below
\fancyfoot[L]{} % Empty left footer
\fancyfoot[C]{} % Empty center footer
\fancyfoot[R]{\thepage} % Page numbering for right footer
\renewcommand{\headrulewidth}{0pt} % Remove header underlines
\renewcommand{\footrulewidth}{0pt} % Remove footer underlines
\setlength{\headheight}{13.6pt} % Customize the height of the header

\setlength\parindent{0pt} % Removes all indentation from paragraphs - comment this line for an assignment with lots of text


\newcommand{\N}{\mathbb{N}}
\newcommand{\Z}{\mathbb{Z}}

\newenvironment{Questions}[1][Question]
{\begin{enumerate}}
	{\end{enumerate}}
\newenvironment{Parts}[1][Part]
{\begin{enumerate}}
	{\end{enumerate}}

\newenvironment{answer}[1][Answer]
{\begin{trivlist}\item[\hskip \labelsep {\bfseries #1:}]}
	{\end{trivlist}}

\newcommand{\Question}{\item }
\newcommand{\Part}{\item}

\newcommand{\horrule}[1]{\rule{\linewidth}{#1}} % Create horizontal rule command with 1 argument of height

\usepackage{color}

%New colors defined below
\definecolor{codegreen}{rgb}{0,0.6,0}
\definecolor{codered}{RGB}{204,0,0}
\definecolor{codeblue}{RGB}{0,128,255}
\definecolor{backcolour}{rgb}{0.95,0.95,0.92}

%Code listing style named "mystyle"
\lstdefinestyle{mystyle}{
  commentstyle=\color{codegreen},
  keywordstyle=\color{codeblue},
  stringstyle=\color{codegreen},
  basicstyle=\footnotesize,
  breakatwhitespace=false,
  breaklines=true,
  captionpos=b,
  keepspaces=true,
  showspaces=false,
  showstringspaces=false,
  showtabs=false,
  tabsize=2
}

%"mystyle" code listing set
\lstset{style=mystyle}

\title{
	\normalfont \normalsize
	\textsc{CSM61B} \\ [25pt] % Your university, school and/or department name(s)
	\horrule{0.5pt} \\[0.4cm] % Thin top horizontal rule
	\huge Asymptotics Practice Problems \\ % The assignment title
	\horrule{2pt} \\[0.5cm] % Thick bottom horizontal rule
}

\author{Alex Kazorian} % Your name

\begin{document}

	\maketitle
	\begin{Questions}
        % Question 1
	    \Question
        Order the following functions so that $f_i \in O(f_j) \Longleftrightarrow i < j$:
            \begin{enumerate}
                \item $f_1(n) = 3^n$
                \item $f_2(n) = \frac n 2$
                \item $f_3(n) = 2^10000$
                \item $f_4(n) = 2^{\log_2 n}$
                \item $f_5(n) = \log n$
                \item $f_6(n) = n + n^2 \log n$
                \item $f_7(n) = n!$
                \item $f_8(n) = 1.001^n + n^3$
            \end{enumerate}

        \Question
        Provide the tightest bound on $f(n)$ in terms of $g(n)$ by saying $f = O(g)$, $f = \Omega(g)$, or $f = \Theta(g)$.
            \begin{enumerate}
                \item $f(n) = \log_2 n$ \\
                      $g(n) = \log_3 n$
                \item $f(n) = \log n^2 \\
                       g(n) = \log n$
                \item $f(n) = n - 100 \\
                       g(n) = n + 10000$
                \item $f(n) = 2^{1.1n} \\
                       g(n) = \sum_i^n i^2$
            \end{enumerate}

        \Question
        Find the tightest bound of the function $f(n) = x \sin x^2$ in both the worst and best case.

        \Question
		Provide the worst and best runtime for the following methods in terms of $N$.
			\begin{enumerate}
				\item
				Here consider $N$ to be the length of the array and $p$ to be initialized to $N$.
				\begin{lstlisting}[language=Java]
public void int foo(int[] fighters, int p) {
	if (!p) {
		return fighters[p];
	}
	if (p % 2 == 0) {
		return foo(fighters, p / 2);
	}
	System.arraycopy(fighters, 0, fighters, 0, p + 1);
	fighters[p] = p;
	return foo(fighters, p + 1);
}
			\end{lstlisting}
				% Answer for the above is worst case: N and best case: log N where N is the size of the input
				\item
			\end{enumerate}

	\end{Questions}




\end{document}
